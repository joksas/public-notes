\documentclass{article}

\usepackage[a4paper,left=20mm, right=20mm, top=20mm, bottom=20mm]{geometry}
\usepackage[onehalfspacing]{setspace}
\usepackage[backend=bibtex, style=nature, sorting=none, doi=true, bibencoding=utf8, minbibnames=4, maxbibnames=4]{biblatex}

\setlength{\parindent}{0em}
\setlength{\parskip}{1em}

\addbibresource{~/.latex-bibliography/miscellaneous.bib}

\newcommand{\prop}[1]{(TLP~#1)}

\begin{document}

\section*{\prop{1}}

\prop{1} says ``The world is everything that is the case.''

\prop{1.1} says ``The world is the totality of facts, not of things.''

The concept of logical space is introduced in \prop{1.13}. This is a space of combinations of objects that are logically possible (though not necessarily true), e.g.\ ``Toronto is the capital of Canada'' is in the logical space, while ``love is purple'' is not \cite{Vi2019}.

Wittgenstein also mentions the independence of facts from one another in \prop{1.21}: ``Any one [fact] can be either the case or not be the case, and everything else remain the same.''

\section*{\prop{2}}

\prop{2} introduces the idea of facts and atomic facts: ``What is the case, the fact, is the existence of atomic facts.''
That is, fact consists of atomic facts.

\prop{2.01} states that ``an atomic fact is a combination of objects (entities, things)''.
So while Bertrand Russell states that every fact contains at least one universal\footnote{property that can be shared by more than one object, e.g.\ color}, Wittgenstein makes no distinction between universals and particulars \cite{Sp2007}.

In \prop{2.011}, Wittgenstein adds that ``it is essential to a thing that it can be a constituent part of an atomic fact''.

Examples in \prop{2.0131} suggest that although objects need not be part of any particular atomic fact, it must be part of \emph{some} atomic fact \cite{Sp}.
``A speck in a visual field need not be red, but it must have a colour; it has, so to speak, a color space round it.
A tone must have \emph{a} pitch, the object of the sense of touch \emph{a} hardness, etc.''

\prop{2.02} states that ``the object is simple''.
\prop{2.021} expands on what it means for an object to be simple: ``Objects form the substance of the world. Therefore they cannot be compound.''
This seems to prevent concepts like tables and chairs to be objects \cite{Vi2019}.
It is thus not very clear if such objects even exist.

Wittgenstein's argument for the existence of things seems to rely on \prop{2.0211} and \prop{2.0212} \cite{Sp2007}: ``If the world had no substance\footnote{[something that is formed of objects according to \prop{2.021}]}, then whether a proposition had sense would depend on whether another proposition was true.''\ and ``It would then be impossible to form a picture of the world (true or false).''
The argument is that without the existence of objects we could not represent the world, and so because we can represent it, it means that objects exist \cite{Sp2007}.
But why are simple entities (objects) necessary for representation?

There exist conflicting interpretations of \prop{2.021}--\prop{2.0212} and whether they contain the proof for the claims of the existence of things and them being a requirement for representation \cite{Lu1976, Gr1964, We1935, An1959, Pi1964}. 
\cite{Lu1976} claims that philosophers often overlook the fact that \prop{2.0211}, \prop{2.0212} do not mention objects at all, and that the concepts of `objects' and `substance' are incorrectly treated as synonymous.
\cite{Sp2007} explores this argument using \prop{3.203} and \prop{4.024}.

\printbibliography[title=References]

\end{document}
